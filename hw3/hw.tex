\documentclass{article}
\usepackage[utf8]{inputenc}
\usepackage[margin = 0.8in]{geometry}
\usepackage{graphicx}
\usepackage{amsmath, amssymb}
\usepackage{subcaption}
\usepackage{multirow}
\usepackage{mathtools}
\usepackage{float}


\title{RBE549 - Homework 3}
\author{Keith Chester}
\date{Due date: September 22, 2022}

\begin{document}
\maketitle

\section*{Problem 1}


\section*{Problem 2}

In this problem, we are tasked with proving that if $f(x)$ is an odd function and purely imaginiary (ie no real component), then the Fourier Transform of $f(x)$, $F(x)$ is both real and odd.

We start with defining $F(x)$:

\begin{equation}
    \int_{-\infty}^{\infty} f(x)e^{-j \omega x} dx
\end{equation}

\noindent ...and we can split this integral into two parts, specifically utilizing that $\int_{-A}^{A}$ can be split into muliple integrals of along the requested range, such that $\int_{-A}^0 + \int_0^A$ is equivalent.

\begin{equation}
    \int_{-\infty}^{0} f(x)e^{-j \omega x} dx + \int_{0}^{\infty} f(x)e^{-j \omega x} dx
\end{equation}

\noindent We can then exapnd this with an idenity:

\begin{equation}
    \int_{-\infty}^{0} f(x)(\cos(\omega x) - j\sin(\omega x)) dx + \int_{0}^{\infty} f(x)(\cos(\omega x) - j\sin(\omega x)) dx
\end{equation}

\noindent Similarly to before, we can utilize a property of integrals to further separate this - specifically that $\int (A(x)+B(X)) = \int A(x) + \int B(x)$.

\begin{equation}
    \int_{-\infty}^{0} f(x)\cos(\omega x) dx + \int_{-\infty}^{0} -j f(x) \sin(\omega x) dx + \int_{0}^{\infty} f(x)\cos(\omega x) dx + \int_{0}^{\infty} -j f(x) \sin(\omega x) dx
\end{equation}

\noindent We can reorder terms to isolate the cosine terms, and move $-j$ (since it's a constant) out of the integral:

\begin{equation}
    \int_{-\infty}^{0} f(x)\cos(\omega x) dx + \int_{0}^{\infty} f(x)\cos(\omega x) dx - j \int_{-\infty}^{0} f(x) \sin(\omega x) dx - j \int_{0}^{\infty} f(x) \sin(\omega x) dx
\end{equation}

\noindent To prove that a function is odd, we must show that $f(-x) = -f(x)$. However, we know that $\cos$ is an even function, meaning $f(-x)=f(x)$. thus if we were to plug in $-x$, our negative sign does not escape the cosine, resulting in a cancellation of terms. This ultimately leads us to:

\begin{equation}
    F(x) = -j \int_{-\infty}^{\infty} f(x) \sin(xt) dt
\end{equation}

\noindent Since $j$ is our imaginary component, and we have a singular component left, it shows that our Fourier transform is in fact completely imaginary. If we let $f(x) = jg(x)$ as $f(x)$ is purely imaginiary:

\begin{equation}
    F(x) = -j \int_{-\infty}^{\infty} jg(x) \sin(xt) dx = -j^2 \int_{-\infty}^{\infty} g(x) \sin(xt) = \int_{-\infty}^{\infty} g(x) \sin(xt)
\end{equation}

This result is an odd function (even or odd functions multiplied by an odd function is odd) with no imaginary components!


\section*{Problem 3}


\section*{Problem 4}
For this problem we look at two kernels that are used to detect diagonal edges:

\begin{equation}
    NE = \begin{bmatrix}
        1 & 1 & 0 \\
        1 & 0 & -1 \\
        0 & -1 & -1
    \end{bmatrix}
\end{equation}

\begin{equation}
    NW = \begin{bmatrix}
        0 & 1 & 1 \\
        -1 & 0 & 1 \\
        -1 & -1 & 0
    \end{bmatrix}
\end{equation}

\subsection*{A}

How are these operates related to Sobel H and Sobel V?

\subsection*{B}

Now we aim to suggest two ways to combine these operators into a singular kernel that can identify northeast or northwest diagonals together, and discuss the potential problems with these combinations.

\subsection*{C}

This problem asks us to express $NW$ as a convolution of an unknown $2x2$ operator with the kernel of:

\begin{equation}
    \begin{bmatrix}
        0 & 1 \\
        -1 & 0 \\
    \end{bmatrix}
\end{equation}

\noindent ...or, if we labeled the above as $h(x)$, find $g(x)$ such that $g(x)*h(x) = NW$. The resulting matrix for $g(x)$ is:

\begin{equation}
    \begin{bmatrix}
        1 & 1 \\
        1 & 1 \\
    \end{bmatrix}
\end{equation}

\section*{Problem 5}

We look at the outcome of convolving a Sobel vertical edge detector ($V$) with a $3x3$ blurring mask ($B$)

\begin{equation}
    V = \begin{bmatrix}
        -1 & 0 & 1 \\
        -2 & 0 & 2 \\
        -1 & 0 & 1 \\
    \end{bmatrix}
\end{equation}

\begin{equation}
    B = \begin{bmatrix}
        2 & 3 & 2 \\
        3 & 4 & 3 \\
        2 & 3 & 2 \\
    \end{bmatrix}
\end{equation}

\subsection*{A}

We ask - what is the result of the combined convolutional mask for applying $V$ first, and then $B$? If we perform the convolution, we get a matrix that is a $5x5$:

\begin{equation}
    \begin{bmatrix}
        -2 & -3 & 0 & 3 & 2 \\
        -7 & -10 & 0 & 10 & 7 \\
        -10 & -14 & 0 & 14 & 10 \\
        -7 & -10 & 0 & 10 & 7 \\
        -2 & -3 & 0 & 3 & 2 \\
    \end{bmatrix}
\end{equation}

\subsection*{B}

We are asked if convolving with $B$ first, then $V$ produces different results than convolving with $V$ first, and then $B$. The answer is \emph{no} - convolution has a commutative property - $f(x)*g(x) = g(x)*f(x)$. Thus we would have the same result we found in part A of this question.

\subsection*{C}

In this part, we're asked if the convolution mask of $V*B$ seperable into a convolution of x-only and y-only masks. The answer here is \emph{no} as well. In order for a convoluton mask to be seperable, the resulting matrix must have a rank of $1$. The rank of our resulting mask is $2$, thus precluding us from finding a seperable convolution.


\end{document}