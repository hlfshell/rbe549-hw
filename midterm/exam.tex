\documentclass{article}
\usepackage[utf8]{inputenc}
\usepackage[margin = 0.8in]{geometry}
\usepackage{graphicx}
\usepackage{amsmath, amssymb}
\usepackage{subcaption}
\usepackage{multirow}
\usepackage{mathtools}
\usepackage{float}


\title{RBE549 - Midterm Exam}
\author{Keith Chester}
\date{Due date: October 16, 2022}

\begin{document}
\maketitle

\section*{Problem 1}

In this problem we are presented with a Hough transform problem, wherein $x$, $y$, $b$, and $m$ are positive or negative real numbers. 2 points are given in $(m,b)$ space by $P_1: (m,b)=(0.5,4)$ and $P_2: (m,b)=(0.9,0)$.

\subsection*{A}

What line $L_1$ in $(m,b)$ space through $P_1$ and $P_2$?

We can set the line equation ($y=mx+b$) for $P_1$ and $P_2$ equal to eachother to solve this.

\begin{equation}
    P_1 = P_2
\end{equation}

\begin{equation}
    0.5x+4 = 0.9x + 0
\end{equation}

\begin{equation}
    4 = 0.4x
\end{equation}

\begin{equation}
    10 = x
\end{equation}

\noindent ...and then we plug in $x$ to solve for $y$:

\begin{equation}
    0.5(10)+4 = y
\end{equation}

\begin{equation}
    y = 9
\end{equation}

\noindent ...thus the equation for $L_1$, a line that passes through points $P_1$ and $P_2$, is $b = -10m + 9$.

\subsection*{B}

The point $P_3$ in $(x,y)$ space that corresponds to line $L_1$ is the point $(10,9)$, as calculated above in part $A$. We can also just deduce it from our solved equation mentioned in $A$ - specifically that since $y=mx+b$ for $b=-mx+y$, and we stated that $b=-10m+9$, we can also deduce that $(x,y)=(10,9)$.

\subsection*{C}

Since it is a horizontal line, which means the slope $m$ must be $m=0$, we now know that the resulting point, $P_4$, must fall along the line we calculated earlier, $L_1$, specifically where $m=0$, or the vertical axis of $(m,b)$ space. Thus we can solve:

\begin{equation}
    b = -10m + 9
\end{equation}

\begin{equation}
    b = -10*0 + 9
\end{equation}

\begin{equation}
    b = 9
\end{equation}

\noindent ...thus we know that $P_4$ will fall on $(m,b)=(0,9)$. This means that the equation for $L_2$ is $y=9$.

\section*{Problem 2}

In this problem we are aiming to discover a structuring element $SE$ such that $II\bigoplus SE =OI$

\subsection*{A}

A $3x3$ structing element SE that satisfies the given $II$ and $OI$ would be:

\begin{equation}
    \begin{tabular}{ | c | c | c | }
        \hline
        1 & 0 & 1 \\
        \hline
        0 & \textbf{0} & 0 \\
        \hline
        1 & 0 & 1 \\
        \hline
    \end{tabular}
\end{equation}

\noindent ...where the central $0$ is the point of origin for the $SE$.

\subsection*{B}

Another structuring element $SE_2$ can satisfy $II\bigoplus SE_2 = II$ but $SE\neq SE_2$, where again the central value (now a $1$) is the point of origin::

\begin{equation}
    \begin{tabular}{ | c | c | c | }
        \hline
        1 & 0 & 1 \\
        \hline
        0 & \textbf{1} & 0 \\
        \hline
        1 & 0 & 1 \\
        \hline
    \end{tabular}
\end{equation}


\section*{Problem 3}

In this problem we are solving equations having to do with image focus and lenses.

\subsection*{A}

We aim to prove that for a thin lens, the image is in focus when:

\begin{equation}
    \frac{1}{-z_0} + \frac{1}{z_c} = \frac{1}{f}
\end{equation}



\subsection*{B}

In this problem we are tasked with finding the number of receptors on which the image of Mars falls while looking up into the night sky. We say that a human eyeball has a radius of $12mm$ and contains roughly $1.5\times10^8$ receptors. We assume that the receptors are uniform across a $160^\circ$ partial sphere in the back of our eye. The planet of Mars has a $4\times10^3km$ radius with an average distance of $2.25\times10^8km$. We use an $f$ focal value equal to the eye's diameter ($24mm$).

To solve this, first we use similar triangles using a lens diagram (similar to the one provided for the question in $3A$); this allows us to determine the size of the image of Mars on our eye.

\begin{equation}
    \frac{z_0}{z_c} = \frac{h_0}{h_1}
\end{equation}

\noindent In this we have $z_0$ as the distance from Mars to our eye, and $h_0$ the width of Mars, and $h_i$ is the size of the image of Mars on our eye. We need to solve for $z_c$ before we can figure out $h_1$, so to do this we look at the equation utilized earlier:

\begin{equation}
    \frac{1}{-z_0}+\frac{1}{z_c}=\frac{1}{f}
\end{equation}

\begin{equation}
    \frac{1}{-2.25\times10^8km}+\frac{1}{z_c}=\frac{1}{24mm}
\end{equation}

\begin{equation}
    \frac{1}{z_c} \approx 83.33 + 4.44\times10^{-9}
\end{equation}

\begin{equation}
    z_c \approx 0.012 \approx 24 mm
\end{equation}

Now we can determine the size of the image of Mars on our eye:

\begin{equation}
    \frac{z_0}{z_c} = \frac{h_0}{h_1}
\end{equation}

\begin{equation}
    \frac{2.25\times10^8km}{12mm} = \frac{8,000km}{h_1}
\end{equation}

\begin{equation}
    h_1 = 8.53\times10^{-7} m
\end{equation}

\noindent ...which is an incredibly tiny size. So let's figure out how many receptors we have per square $mm$ of our eye. First we need to find the area of a spherical cap, such that we can determine the average number of receptors per mm in the eye.

\end{document}
