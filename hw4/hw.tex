\documentclass{article}
\usepackage[utf8]{inputenc}
\usepackage[margin = 0.8in]{geometry}
\usepackage{graphicx}
\usepackage{amsmath, amssymb}
\usepackage{subcaption}
\usepackage{multirow}
\usepackage{mathtools}
\usepackage{float}


\title{RBE549 - Homework 4}
\author{Keith Chester}
\date{Due date: September 29, 2022}

\begin{document}
\maketitle

\section*{Problem 0}


\section*{Problem 1}


\section*{Problem 2}

In this problem, we are looking at a Hough Transform problem where $x$, $y$, $b$, $c$, $m$, and $n$ may be positive or negative real numbers. $2$ points in $(x,y)$ space are given by $P_1=(2,4)$ and $P_2=(4,3)$.

\subsection*{A}

First, we are tasked with finding $L_1$ and $L_2$, the lines assoicated in $(m,b)$ space corresponding to $P_1$ and $P_2$.

For this we must first solve for $m$ and $b$ to form lines for $(m,b)$ space. Given that $y = mx + b$, we have:

\begin{equation}
    L_1: 4 = 2m + b
\end{equation}

\begin{equation}
    L_1: b = -2m + 4
\end{equation}

\begin{equation}
    L_2: 3 = 4m + b
\end{equation}

\begin{equation}
    L_2: b = -4m + 3
\end{equation}

We can thus plot the lines in $(m,b)$ space from $P_1$ and $P_2$ as such:

TODO: PLOT

\subsection*{B}

Next we are tasked with finding the intersection of these lines. Thus we can set $L_1=L_2$ and find:

\begin{equation}
    -2m + 4 = -4m + 3
\end{equation}

\begin{equation}
    1 = -2m
\end{equation}

\begin{equation}
    m = -\frac{1}{2}
\end{equation}

\noindent ...and then solve for $b$ using our discovered $m$:

\begin{equation}
    b = -2(-\frac{1}{2}) + 4 = 5
\end{equation}

\noindent Thus we find a point in $(m,b)$ space such that $L_1=L_2$ at $(-\frac{1}{2}, 5)$.

\subsection*{C}

In this section, we are then asked what line connects both $P_1$ and $P_2$. This line is what we discovered in part B, wherein we found the the intersection of $L_1$ and $L_2$: $y = -\frac{1}{2}x + 5$.

\subsection*{D}

We are tasked on finding where $L_3$, which passes through $(m,b)$ space of $(0,0)$ and find the coresponding $P_3$ with it. For this we solve the intersection along the $b$ axis. Since $b = -mx + y$, we get $0 = -0x + y$ and thus $y = 0$.

Now that we know $y=0$, we can solve for x:

\begin{equation}
    0 = -\frac{1}{2}x + 5
\end{equation}

\begin{equation}
    x = 10
\end{equation}

\noindent Now that we have the point $P_3=(10,0)$ we can translate this back to $(m,b)$ space:

\begin{equation}
    b = -mx + y
\end{equation}

\begin{equation}
    b = -10m + 0
\end{equation}

\noindent ...which is our resulting $L_3$.

\end{document}