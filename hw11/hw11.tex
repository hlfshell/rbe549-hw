\documentclass{article}
\usepackage[utf8]{inputenc}
\usepackage[margin = 0.8in]{geometry}
\usepackage{graphicx}
\usepackage{amsmath, amssymb}
\usepackage{subcaption}
\usepackage{multirow}
\usepackage{mathtools}
\usepackage{float}


\title{RBE549 - Homework 10}
\author{Keith Chester}
\date{Due date: November 23, 2022}

\begin{document}
\maketitle

\section*{Problem 1}

In this problem ,we are asked to consider two surfaces:

\begin{equation}
    z_1 = \frac{1}{2} \ln(x^2 + y^2)
\end{equation}

\noindent ...and...

\begin{equation}
    z_2 = \tan^-1(\frac{x}{y})
\end{equation}

\subsection*{A}

We are asked to find $p(x,y)$ and $q(x,y)$ for both surfaces. $p(x,y)=\frac{\partial z_n}{\partial x}$ and $q(x,y)=\frac{\partial z_n}{d_y}$. So we solve for each:

\begin{equation}
    p_1 = \frac{\partial z_1}{\partial x} = \frac{\partial}{\partial x} \frac{1}{2} \ln(x^2 + y^2) = \frac{x}{x^2+y^2}
\end{equation}

\begin{equation}
    q_1 = \frac{\partial z_1}{\partial y} = \frac{\partial}{\partial y} \frac{1}{2} \ln(x^2 + y^2) = \frac{y}{x^2+y^2}
\end{equation}

\begin{equation}
    p_2 = \frac{\partial z_2}{\partial x} = \frac{\partial}{\partial x} \tan^-1(\frac{x}{y}) = \frac{y}{x^2 + y^2}
\end{equation}

\begin{equation}
    q_2 = \frac{\partial z_2}{\partial y} = \frac{\partial}{\partial y} \tan^-1(\frac{x}{y}) =  -\frac{x}{x^2+y^2}
\end{equation}

\subsection*{B}

Here, we are tasked in showing that $z_1$ and $z_2$ gives rise to the same shading when a rotationally symmetric reflectance map applies; $R(p,q) = R(p^2 + q^2)$.

Here, we can show this by demonstrating that $p_1^2 + q_1^2 = p_2^2 + q^2$.

\begin{equation}
    p_1^2 + q_1^2 = p_2^2 + q^2
\end{equation}

\begin{equation}
    \frac{x}{x^2+y^2}^2 + \frac{y}{x^2+y^2}^2 = \frac{y}{x^2 + y^2}^2 + -\frac{x}{x^2+y^2}^2
\end{equation}

\begin{equation}
    \frac{x^2}{x^2+y^2} + \frac{y^2}{(x^2+y^2)^2} = \frac{y^2}{(x^2+y^2)^2} + \frac{x^2}{x^2+y^2}
\end{equation}

\begin{equation}
    \frac{1}{x^2+y^2} = \frac{1}{x^2+y^2}
\end{equation}

\noindent


\section{Problem 2}

In this problem, we're looking at a surface normal $\hat{n}=\begin{bmatrix}n_x & n_y & n_z\end{bmatrix}^T$, where $n_x^2 + n_y^2 + n_z^2 = 1$, and can be represented by surface orientation $p,q.p = \frac{n_x}{\sqrt{1-n_x^2-n_y^2}}$, etc. In addition to the original origin, $\begin{bmatrix}0 & 0 & 0 \end{bmatrix}^T$, we can use another origin from the point $\begin{bmatrix}0 & 0 & -1 \end{bmatrix}^T$.

\subsection*{A}

Here, we are tasked with solving for $f$ and $g$. We can do this by solving for the line of the vector from our new origin point of $\begin{bmatrix}0 & 0 & -1 \end{bmatrix}^T$.

\begin{equation}
    \begin{bmatrix}
        x \\ y \\ z
    \end{bmatrix} = \begin{bmatrix}
        a \\ b \\ c 
    \end{bmatrix} + k \begin{bmatrix}
        l \\ m \\ n
    \end{bmatrix}
\end{equation}

\noindent Note that the slopes represented by our variables can be further defined as:

\begin{equation}
    \begin{bmatrix}
        l \\ m \\ n
    \end{bmatrix} = \begin{bmatrix}
        x_2 - x_1 \\ y_2 - y_1 \\ z_2 - z_1
    \end{bmatrix}
\end{equation}

\noindent Our $x$ and our $y$ are the values we're seeking to solve for, $f$ and $g$, respectively. Now we fill in our known values to solve for our line:

\begin{equation}
    \begin{bmatrix}
        x \\ y \\ z
    \end{bmatrix} = \begin{bmatrix}
        a \\ b \\ c 
    \end{bmatrix} + k \begin{bmatrix}
        x_2 - x_1 \\ y_2 - y_1 \\ z_2 - z_1
    \end{bmatrix}
\end{equation}

\begin{equation}
    \begin{bmatrix}
        f \\ g \\ 1
    \end{bmatrix} = \begin{bmatrix}
        0 \\ 0 \\ -1 
    \end{bmatrix} + k \begin{bmatrix}
        n_x - 0 \\ n_y - 0 \\ n_z + 1
    \end{bmatrix}
\end{equation}

\noindent We can then break this down into a system of equations. First, we solve for $t$ using our $z$ value as we have an additional known here.

\begin{equation}
    1 = -1 + k (n_z + 1)
\end{equation}

\begin{equation}
    k = \frac{2}{n_z+1}
\end{equation}

\noindent ...and now we can solve for $f$ and $g$:

\begin{equation}
    f = 0 + k (n_x-0) = \frac{2n_x}{n_z+1}
\end{equation}

\begin{equation}
    g = 0 + k(n_y-0) = \frac{2n}{n_z+1}
\end{equation}

\noindent Resulting in a final $f$, $g$, $z$ = 

\begin{equation}
    \begin{bmatrix}
        f \\ g \\ 1
    \end{bmatrix} = \begin{bmatrix}
        \frac{2n_x}{n_z+1} \\ \\ \frac{2n_y}{n_z+1} \\ \\ 1
    \end{bmatrix}
\end{equation}

\subsection*{B}

Now we are taskedwith showing that when $n_z = 0$, $f$ and $g$ are finite and lie on a circle of radius $2$.

\begin{equation}
    f = \frac{2n_x}{n_z+1} = \frac{2n_x}{0+1} = 2n_x
\end{equation}

\begin{equation}
    g = \frac{2n}{n_z+1} = \frac{2n_y}{0+1} = 2n_y
\end{equation}

\noindent Given these values, we can confirm the statement that $f$ and $g$ are finite as they depend upon $n_x$ and $n_y$ respectively. Now, the radius is the result of the magnitude of the vector. This means we can do the following:

\begin{equation}
    |f,g| = \sqrt{f^2 + g^2} = \sqrt{ (2n_x)^2 + (2n_y)^2 } = \sqrt{4(\sqrt{1-n^2_y})^2 + (2n_y)^2} = \sqrt{4 - 4n_y^2 + 4n_y^2} = \sqrt{4} = 2
\end{equation}


\end{document}